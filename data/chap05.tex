% !TeX root = ../thuthesis-example.tex

\chapter{THEORETICAL EVALUATION OF VR-IOT RESEARCH PLATFORM}
В данной главе будет приведена теоретическая оценка различных параметров платформы. В отличие от существующих решений на рынке, это - первая платформа, позволяющая автоматически переносить практически любые устройства из реального мира в виртуальный. Помимо этого, платформа также позволяет создавать новые устройства в виртуальной реальности посредством конструктора устройств, о котором будет рассказано позднее. Несмотря на то, что основная цель работы состояла в проектривании платформы, будет также показано, как должен выглядеть API для внешних проектов, подключенных к платформе.

Если в предыдущей части мы рассматривали ограничения, продиктованные железом, то в этой части мы сосредоточимся именно на ограничениях, продиктованных архитектурой, а также покажем, что на самом деле их нет.

Во-первых, будет доказан полнота поддержки устройств. Под полнотой здесь подразумевается поддержка практически всех существующих стандартов передачи (перечисляю), а также значений items. Я перечисляю различные типы параметров для реальных устройств и показываю, что практически все они могут быть распарсены уже на этапе хранения на сервере. Пример: различные сенсоры, возвращающие или число, или булевое значение. Далее передача голоса, также доступна, передача различных команд идет через string command, передача видео доступна.

То есть, основываясь на том, что опенхаб поддерживает практически любые типы устройств, мы, посредством перенесения структуры опенхаба на нуих студию достиглы той же полноты поддержки. Тем не менее, необходимо не только уметь визуализировать айтемс, но еще и изменять их в вр. И вот тут кроется основная задача разработки. Тут я могу сослаться на версию 0.5 платформы, где были реализованы различные типы устройств, а также теоритически расписать, поддержка каких сенсоров будет доступна.

Для создания новых устройств используется конуструктор. На самом устройстве можно создавать новые IoT устройства внутри виртуальной реальности, и они автоматически добавляются на сервер, появляясь в окружении реальных умных устройств. Далее данные устройства можно модернизировать, а также проводить дополнительные связи чтоооо

В общем, потом показываю, что ограничения, продиктованные физикой, могут быть реализованы на сервере, как это сказано в предыдущей главе, но тем не менее, сначала надо разработать данную физику, так как движук юнити не настолько продвинутый.

Тем не менее, как практически, так и теоретически показано, что платформа может использоваться для создания новых устройств.
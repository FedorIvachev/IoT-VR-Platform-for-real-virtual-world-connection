% !TeX root = ../thuthesis-example.tex

% 中英文摘要和关键字
% Abstract in CHINESE
% Requirements: 800-1000 CHINESE characters, it should not exceed 1 page maximum. You are allowed a maximum of 5 keywords in total
\begin{abstract}

近年来,计算机处理数据的成本变得愈发廉价,这使得我们令周围事物变得智能化不再成为问题:在服饰、家具、玩具等物品中使用节能的系统级芯片或微控制器,助力于对我们活动监控的实施,也帮助储存传感器数据,对其进行分析并共享到互联网。这些智能的发明可以通过使用嵌入式麦克风,扬声器和信息显示器与我们进行互动。 5G和6G网络的快速发展,有可能在不久的将来为人们提供一种快速且节能的方式来对成千上万台设备进行连接,而不再是仅仅依附于某一个地点。设备之间更高数据的共享以及对大数据分析的改进提升,促进着物联网的进一步发展。由于当下缺乏能够使它们快速集成到现实世界现有场景当中的标准软件,开发新型设备成为了一个问题。

同时,虚拟现实技术近来得到了飞速地发展,并且伴随着新的手部识别技术以及虚拟现实头戴显示器设备的革新,它已证明自己不仅仅是用于游戏目的,而且还得以应用于医疗保健,教育,和科学。

作者在此提出了一种同步机制,该机制可以根据虚拟世界中的动作和事件使真实和虚拟设备持续更新,反之亦然 —— 将应用于真实设备中的动作转移到虚拟世界中去。在本文中,人们或许会质疑,是否可以借助于虚拟现实交互设备开发出这种机制并将其用作为新设备研究方式的一部分,以及是否有可能通过以下方式扩展现实世界的设备 —— 额外的虚拟传感器和交互式对象。
  
本文所研究提出的即是一种适合于该机制的体系结构。该体系结构通过搭建一个原型平台以实现真实世界与虚拟世界之间的无缝对接。借力于各物联网的一个虚拟代表,设备参数(数据值)以及双向数据将在真实世界与虚拟世界之间进行传输,同时系统的性能将会被评估,从而为日后的开发提供数据支撑,也为将来的改进提供参考。本文还讨论了如何调整平台,以使得其能够在混合现实和增强现实设备中运行。尽管该平台目前仍处于原型阶段,但它已被应用至清华大学学生的项目实践当中。

该平台可在同一个IoT-VR环境中为多个用户进行同步工作提供可能,此外还可提供先进的物理计算和数据分析,以及其他解决方案的集成。



  % 关键词用“英文逗号”分隔,输出时会自动处理为正确的分隔符
  % Input your keywords in CHINESE, separated by a "english comma" to ensure correct formatting. 
  \thusetup{
    keywords = {物联网, 虚拟现实, 人机交互},
  }
\end{abstract}

% Corresponding abstract in English
% Note that the keywords and the contents of the abstract should match with the CHINESE version
\begin{abstract*}

  In recent years processing power has become so cheap that making objects around us intelligent is no longer an issue: using energy-effective System-on-Chips or microcontrollers in clothes, furniture, toys, and other things helps to monitor our activity, store sensors' data, analyze it and share it to the Internet. Such smart things can interact with us using the embedded microphones, speakers and information displays. The rapid development of 5G- and 6G-networks in the near future will make it possible to provide a fast and energy-effective way to connect thousands of devices, no longer attached to one location. Higher data sharing between the devices and improved Big Data analysis leads to further evolution of Internet of Things. Developing new types of devices is an issue since there is no standard software for  their fast integration into real-world existing scenarios.
   
  In the meantime, Virtual Reality has seen rapid development recently, and with the newer techniques of hands recognition, as well as the evolution of Virtual Reality Headsets, it has proven itself as a great instrument not only for gaming purposes but for healthcare, education, and science.
   
  The author proposes a synchronization mechanism that keeps the real and virtual devices up-to-date, according to actions and events in the virtual world and vice-versa: actions applied on the real devices are transferred to the virtual world. In this dissertation, it is questioned if, with the help of Virtual Reality interaction devices, such mechanism can be developed and be used as a part of an instrument for research on new devices, and whether it is possible to extend real-world devices with extra virtual sensors and interactive objects.
  
  In this dissertation, a suitable architecture for this mechanism is proposed. By creating a prototype of the platform for seamless connection between real and virtual world, with a virtual representation of each Internet Of Things device parameters (data values) and two-way data transfer between real and virtual world, the performance of the system is evaluated, providing results for future development and consideration for future improvements. It is also discussed how to adapt the platform for running on Mixed reality and Augmented reality devices. Although the platform is in the prototype stage, it has already been used by the students of Tsinghua University in their course projects. 
  
  The platform can provide simultaneous work of multiple users within the same IoT-VR environment, advanced physics calculation and data analysis, as well as the integration of other solutions.
 


  % Use comma as seperator when inputting
  \thusetup{
    keywords* = {IoT, VR, HCI},
  }
\end{abstract*}

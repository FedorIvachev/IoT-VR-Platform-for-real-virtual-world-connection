% !TeX root = ../thuthesis-example.tex

% 中英文摘要和关键字
% Abstract in CHINESE
% Requirements: 800-1000 CHINESE characters, it should not exceed 1 page maximum. You are allowed a maximum of 5 keywords in total
\begin{abstract}
  Requirements: 800-1000 CHINESE characters, it should not exceed 1 page maximum. You are allowed a maximum of 5 keywords in total.
  
  论文的摘要是对论文研究内容和成果的高度概括。
  摘要应对论文所研究的问题及其研究目的进行描述,对研究方法和过程进行简单介绍,对研究成果和所得结论进行概括。
  摘要应具有独立性和自明性,其内容应包含与论文全文同等量的主要信息。
  使读者即使不阅读全文,通过摘要就能了解论文的总体内容和主要成果。

  论文摘要的书写应力求精确、简明。
  切忌写成对论文书写内容进行提要的形式,尤其要避免“第 1 章……;第 2 章……;……”这种或类似的陈述方式。

  关键词是为了文献标引工作、用以表示全文主要内容信息的单词或术语。
  关键词不超过 5 个,每个关键词中间用分号分隔。

  % 关键词用“英文逗号”分隔,输出时会自动处理为正确的分隔符
  % Input your keywords in CHINESE, separated by a "english comma" to ensure correct formatting. 
  \thusetup{
    keywords = {关键词 1(keyword 1 in CHINESE), 关键词 2(K2 IN CHINESE), 关键词 3, 关键词 4, 关键词 5},
  }
\end{abstract}

% Corresponding abstract in English
% Note that the keywords and the contents of the abstract should match with the CHINESE version
\begin{abstract*}

  In recent years processing power has become so cheap that making objects around us intelligent is no longer an issue: using energy-effective System-on-Chips or microcontrollers put into clothes, furniture, toys, and other things helps to monitor our activity, store sensors' data, analyze it and share it to the Internet. Such smart things can interact with us using the embedded microphones and speakers. The rapid development of 5G- and 6G-networks in the near future makes it possible to provide a fast and energy-effective way to connect thousands of devices, no longer attached to one location. Higher data sharing between the devices and improved Big Data analysis leads to further evolution of Internet of Things. Developing new types of devices is an issue since there is no standard software for fast integrating ones into real-world existing scenarios.
   
  On the other hand, Virtual Reality has been growing recently, and with the newer techniques of hands recognition, as well as the evolution of Virtual Reality Headsets, it has proven itself as a great instrument not only for gaming purposes but for healthcare, education, and science. The biggest IT companies in the world invest in developing Virtual and Augmented Reality research.
   
  The author proposes a synchronization mechanism that keeps the real and virtual devices up-to-date, according to actions and events in the virtual world and vice-versa: actions applied on the real devices are transferred to the virtual world. In this dissertation, it is questioned if, with the help of Virtual Reality interaction devices, such mechanism can be developed and be used as a part of the instrument for research on new devices, and whether it is possible to extend real-world devices with extra virtual sensors and interactive objects.
  
  In this dissertation, a suitable architecture for this platform is proposed. By creating a prototype for seamless connection between real and virtual world, with a virtual representation of each Internet Of Things device parameters (data values) and two-way data transfer between real and virtual world, the performance of the system is evaluated, providing results for future development and consideration for future improvements of the platform. Although the platform is in the prototype stage, it has already been used by students of Tsinghua University for their projects, with selected projects described in the appendix. 
  
  (next 100 words: summarization of results)
 


  % Use comma as seperator when inputting
  \thusetup{
    keywords* = {IoT, VR, HCI, Smart environments, Integration},
  }
\end{abstract*}

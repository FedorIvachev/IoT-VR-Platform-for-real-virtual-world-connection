% !TeX root = ../thuthesis-example.tex

% 中英文摘要和关键字
% Abstract in CHINESE
% Requirements: 800-1000 CHINESE characters, it should not exceed 1 page maximum. You are allowed a maximum of 5 keywords in total
\begin{abstract}
  Requirements: 800-1000 CHINESE characters, it should not exceed 1 page maximum. You are allowed a maximum of 5 keywords in total.
  
  论文的摘要是对论文研究内容和成果的高度概括。
  摘要应对论文所研究的问题及其研究目的进行描述,对研究方法和过程进行简单介绍,对研究成果和所得结论进行概括。
  摘要应具有独立性和自明性,其内容应包含与论文全文同等量的主要信息。
  使读者即使不阅读全文,通过摘要就能了解论文的总体内容和主要成果。

  论文摘要的书写应力求精确、简明。
  切忌写成对论文书写内容进行提要的形式,尤其要避免“第 1 章……;第 2 章……;……”这种或类似的陈述方式。

  关键词是为了文献标引工作、用以表示全文主要内容信息的单词或术语。
  关键词不超过 5 个,每个关键词中间用分号分隔。

  % 关键词用“英文逗号”分隔,输出时会自动处理为正确的分隔符
  % Input your keywords in CHINESE, separated by a "english comma" to ensure correct formatting. 
  \thusetup{
    keywords = {关键词 1(keyword 1 in CHINESE), 关键词 2(K2 IN CHINESE), 关键词 3, 关键词 4, 关键词 5},
  }
\end{abstract}

% Corresponding abstract in English
% Note that the keywords and the contents of the abstract should match with the CHINESE version
\begin{abstract*}
  Corresponding abstract in English
  ***Note that the keywords and the contents of the abstract should match with the CHINESE version
  
  
  An abstract of a dissertation is a summary and extraction of research work and contributions.
  Included in an abstract should be description of research topic and research objective, brief introduction to methodology and research process, and summarization of conclusion and contributions of the research.
  An abstract should be characterized by independence and clarity and carry identical information with the dissertation.
  It should be such that the general idea and major contributions of the dissertation are conveyed without reading the dissertation.

  An abstract should be concise and to the point.
  It is a misunderstanding to make an abstract an outline of the dissertation and words “the first chapter”, “the second chapter” and the like should be avoided in the abstract.

  Keywords are terms used in a dissertation for indexing, reflecting core information of the dissertation.
  An abstract may contain a maximum of 5 keywords, with semi-colons used in between to separate one another.
  
  This is the initial version of abstract.
 
  
  Virtual reality has been growing recently and with the newer techniques of hands recognition, as well as evolution in Head Mounted Displays, has proven itself as a great instrument not only for gaming purposes, but for healthcare, education and science. For IoT, which is another hot topic, new devices are going to be created, based on 5G networks and on 6G networks in the near future. With the evolution of IoT devices connectivity, with higher data sharing between the devices and improved big data analysis, developing new types of devices is an issue, since nowadays there is no instrument for fast integration of such devices into real-world existing scenarios. Author proposes a synchronisation mechanism that keeps the real devices up-to-date, according to actions and events that happened in the virtual world and vice-versa: actions applied on the real devices are transferred to the virtual world. In this dissertation it is questioned if with the help of Virtual Reality interaction devices such mechanism can be developed and be used as a part of the instrument used fir research on new devices, and whether it is possible to real-world devices with virtual items.
  
  In this dissertation, a suitable architecture for this platform is proposed. By creating a prototype for seamless connection between real and virtual world, with virtual representation of each IoT device parameters (data values) and two-way data transfer between real and virtual world, the performance of the system is evaluated, providing results for future development and consideration for future improvements of the platform. Although the platform is in prototype stage, it has been already used by students of Tsinghua University for their projects, with selected projects described in the appendix. 
  
  (next 300 words: summarization of results)
 


  % Use comma as seperator when inputting
  \thusetup{
    keywords* = {IoT, VR, HCI, Smart environments, Integration},
  }
\end{abstract*}

% !TeX root = ../thuthesis-example.tex

\chapter{MATHEMATICAL SYMBOLS AND FORMULA}

\section{Symbols}

As required by the national formatting standards, the template uses \pkg{unicode-math} for formatting mathematical symbols, which differs to the default used by LATEX. 

\begin{enumerate}
  \item Capital Greek letters are italic by default e.g. \cs{Delta}:$\Delta$。
  \item Increment symbol: $\increment$(U+2206)
  \item Vectors, matrices, tensors should be ITALIC, and BOLD using the \cs{symbf} command; for example: $\symbf{A}$, $\symbf{\alpha}$.
  \item For constants and special functions, use the \cs{symup} command. For example:
    $\symup{\pi} = 3.14\dots$; $\symup{e} = 2.718\dots$,
  \item Example of integrals and differentials: $\int f(x) \dif x$。
\end{enumerate}

For more usage of symbols, you could use the following references
\href{http://mirrors.ctan.org/macros/latex/contrib/unicode-math/unicode-math.pdf}{\pkg{unicode-math}}
\href{http://mirrors.ctan.org/macros/latex/contrib/unicode-math/unimath-symbols.pdf}{\pkg{unimath-symbols}}.

For units and metrics, it is recommended to use the following package:
\href{http://mirrors.ctan.org/macros/latex/contrib/siunitx/siunitx.pdf}{\pkg{siunitx}}
It can conveniently handle the space between Greek letters/numbers and the unit. For example:
\SI{6.4e6}{m},
\SI{9}{\micro\meter},
\si{kg.m.s^{-1}},
\SIrange{10}{20}{\degreeCelsius}。



\section{Mathematical formula}

You can use the following environments \env{equation} 和 \env{equation*} for mathematical formulae.

Please pay attention that round brackets should be included before and after referencing an equation: e.g. Equation \eqref{eq:example}.

\begin{equation}
  \frac{1}{2 \symup{\pi} \symup{i}} \int_\gamma f = \sum_{k=1}^m n(\gamma; a_k) \mathscr{R}(f; a_k)
  \label{eq:example}
\end{equation}

When there are multiple equations, please try to align them at the "equal" sign if possible. We recommend using the \env{align} environment.
For example:
\begin{align}
  a & = b + c + d + e \\
    & = f + g
\end{align}


\section{Mathematical axioms and proofs}

You can use \pkg{amsthm} or \pkg{ntheorem} packages to set up your axiom. After loading one of those package, the template will automatically setup the environments: \env{theorem} and \env{proof}

An example:
\begin{theorem}[Lindeberg--Lévy Central Limit Theorem]
  Set random variables $X_1, X_2, \dots, X_n$ i.i.d., with expectation mean of $\mu$ and variance $\sigma^2 \ne 0$. By formula, $\bar{X}_n = \frac{1}{n} \sum_{i+1}^n X_i$, we have
  \begin{equation}
    \lim_{n \to \infty} P \left(\frac{\sqrt{n} \left( \bar{X}_n - \mu \right)}{\sigma} \le z \right) = \Phi(z),
  \end{equation}
  where $\Phi(z)$ is the function of a normal distribution.
\end{theorem}
\begin{proof}
  Trivial.
\end{proof}

The template also provides the following environments: \env{assumption}、\env{definition}、\env{proposition}、
\env{lemma}、\env{theorem}、\env{axiom}、\env{corollary}、\env{exercise}、
\env{example}、\env{remar}、\env{problem}、\env{conjecture} 
% !TeX root = ../thuthesis-example.tex

% Input your chapter title here
\chapter{LITERATURE REVIEW}

\cite{hu_virtual_2021}
\cite{ahmad_software_2021}
\cite{alam_augmented_2017}
\cite{ankireddy_augmented_2019}
\cite{carneiro_bim_2018}
\cite{chakareski_uav-iot_2019}

\section{Figures}

For insertion of figures, we use \cs{includegraphics} under the \env{figure} environment. Below is an example code for including and referencing a figure: Figure~\ref{fig:example}.

We suggest that you use PDF for vector graphics, e.g. visualizations of data; JPG for photos or images; and PNG for others.

NOTE: TIFF or EPS are not supported.

% Example code for including a figure
\begin{figure}
  \centering
  \includegraphics[width=0.6\linewidth]{example-image-a.pdf}
  \caption{Please input your figure title}
  {Please enter your figure captions here}
  \label{fig:example}
\end{figure}

If there are more than one or more sections to the figure, each panel would be labeled by (a), (b), c()...Additionally, each panel should have a subtitle.

Below is an example: Figure~\ref{fig:subfig-a} and Figure~\ref{fig:subfig-b}


\begin{figure}
  \centering
  \subcaptionbox{Subfigure A title\label{fig:subfig-a}}
    {\includegraphics[width=0.45\linewidth]{example-image-a.pdf}}
  \subcaptionbox{Subfigure B title\label{fig:subfig-b}}
    {\includegraphics[width=0.45\linewidth]{example-image-b.pdf}}
  \caption{Please input your title for this figure}
  {Please enter your figure captions here}
  \label{fig:multi-image}
\end{figure}



\section{Tables}

Tables should be clear and easy to read. Use three-line tables if possible, for example Table~\ref{tab:three-line}.

The three-lines in the table can be generated by the package \pkg{booktabs}.

\begin{table}
  \centering
  \caption{Please input your title here for the three-line table}
  \begin{tabular}{ll}
    \toprule
    File name          & Description               \\
    \midrule
    thuthesis.dtx   & Source file for the template \\
    thuthesis.cls   & Template file                \\
    thuthesis-*.bst & BibTeX Reference file        \\
    thuthesis-*.bbx & BibLaTeX Reference sample file \\
   
    \bottomrule
  \end{tabular}
  \label{tab:three-line}
\end{table}

If there are necessary superscripts for the table contents, you can use \pkg{threeparttable}. The lettering would be a, b, c, etc...

Below is an example:

\begin{table}
  \centering
  \begin{threeparttable}[c]
    \caption{Example for table with superscripts}
    \label{tab:three-part-table}
    \begin{tabular}{ll}
      \toprule
      文件名                 & 描述                         \\
      \midrule
      thuthesis.dtx\tnote{a} & Template source file \\
      thuthesis.cls\tnote{b} & Template file        \\
      thuthesis-*.bst        & BibTeX Reference template file                                 \\
      thuthesis-*.bbx        & BibLaTeX Reference File                                               \\
      thuthesis-*.cbx        & BibLaTeX Example     \\
      \bottomrule
    \end{tabular}
    \begin{tablenotes}
      \item [a] Description of note a,
      \item [b] Description of note b.
    \end{tablenotes}
  \end{threeparttable}
\end{table}

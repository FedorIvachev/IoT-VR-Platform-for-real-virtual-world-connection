% !TeX root = ../thuthesis-example.tex

\chapter{SUMMARY, DISCUSSION AND FUTURE DIRECTIONS}

Итоги:

Цель работы выполнена: представлена платформа для создания новых устройств для существующего иот мира в виртуальной реальности, помогающая создавать новые устройства. Показано, что она позволяет это выполнить как теоретически, так и практически.

Оригинальность работы состоит в том, что не существует исследований на тему создания и взаимодействия устройств для VR-IoT окружения, все исследователя концентрируются именно на управлении IoT устройствами с помощью шлемов виртуальной реальности.

Тем не менее, предстоит еще долгий путь. Тут я могу привести возможный таймлайн по дальнейшей разработке, и сказать, что за год получилось разработать довольно много, несмотря на ограничения, продиктованные расстоянием, и сложностью провести полноценный user study. Тем не менее, расписываю конкретные планы, вязв из из презентации, которые скинул янлину. Дальше повторяю еще раз, что планы, скорее всего, поменяются. Тем не менее, можно на них частично опираться.



В отличие от существующих решений на рынке, это - первая платформа, позволяющая автоматически переносить практически любые устройства из реального мира в виртуальный. Помимо этого, платформа также позволяет создавать новые устройства в виртуальной реальности посредством конструктора устройств

